\section{Related Work}
\label{sec_related_work}

% TODO
% IMDb und Freebase, wurden bislang noch nicht erwähnt - lange Form

In the context of the LOD Triplification Challenge\footnote{\url{http://triplify.org/Challenge/}} Hassanzadeh and Consens developed a Linked Movie Database \cite{LMDB}.
They tried to unify multiple sources of movie data (among them the Internet Movie Database (IMDb) and Freebase) into an open linked data set.
They utilize approximate string matching methods on movies' titles to find sameAs relation between movies of different sources.
Hassanzadeh and Consens measured the accuracy of their method when using edit similarity, Jaccard or WeightedJaccard, as well as other measures used in information retrieval for string matching.
In contrast, the approach presented in this paper uses much more than just the movie titles to improve the sameAs matching results.

Another approach to find sameAs matches is the automated detection of defining features. 
Such an approach was developed by Thalhammer et al. in \cite{MovieSummarization}.
They propose to find features which are important for identifying a certain entity in a linked data set.
Thalhammer et al. do this to allow a summarization of an entity according to only some of its features which is then used to measure similarity to other entities.
They focus on the movie domain for their example similarity measures.

Creating linked open datasets and interlinking between the entities is a problem in many domains. 
There have been multiple attempts to solve it, such as \cite{SensorData} who worked with sensor data. 
Le-Phouc at al. created a live linked open database with various semantic enriched sensor data.
Whereas \cite{openDrugData} have focused on the medical domain and created a dataset of drugs, traditional Chinese medicine, clinical trials, diseases, and pharmaceutical companies.
\cite{osCommerce} triplified an entire open source online shop, osCommerce, to allow search engines to find the structured data behind this system.  
Triplification has even been employed in the sports domain.
One such example is \cite{smm} which generated a live linked open dataset of soccer data combining different data sources, e.g. Fußballdaten.de and uefa.com as well as DBpedia.


%\cite{linkedData_DesignIssues} defines four rules to properly defining linked open data.
%Berners-Lee proposes to use URIs as names for things, and to use HTTP URIs to allow people to look these up.
%Additionally, Berners-Lee states that useful information using certain standards should be provided and that other links to other URIs should be included.


%Triplification (http://linkedmdb.org/), Referenz zur Triplification  other winners of triplification challenge:
%Live linked open sensor database - http://dl.acm.org/citation.cfm?id=1839763
%http://triplify.org/Challenge/Nominations



%Oktie Hassanzadeh and Mariano Consens. Linked Movie Database
%\url{http://ceur-ws.org/Vol-538/ldow2009_paper12.pdf}



%Leveraging Usage Data for Linked Data Movie Entity Summarization
%\url{http://arxiv.org/abs/1204.2718}



