\subsection{Matching}
\label{subsec_evaluation_matching}

\subsubsection{Introduction}
The matching algorithm, explained in \ref{subsec_method_matching}, depends on multiple parameters. The following paragraphs explain the process of tuning these parameters and evaluating the quality of the matching algorithm.

Evaluation is done on a set of 4000 randomly selected movies. The selection is done by listing all movies by name and then using $scala.util.Random$ to shuffle the list and take the first 4000 movies. To get a gold standard we only try to match those movies with an annotated IMDb-Id (but not using the Id to match the movie) and only the movies that could potentially be matched, i.e movies that are in our existing database. The result is a list of 2028 movies that could possibly be matched and are annotated with a gold standard.

Using the algorithm explained in \ref{subsec_method_matching} the following three sets exists:

\begin{description}
\item[TP]: the correctly matched movies,
\item[FP]: the incorrectly matched movies,
\item[FN]: the not matched movies.
\end{description}

Based on these the evaluation measures are defined as follows:



what is evaluated TODO
enviroment TODO
explanation of Precision, Recall, FMeasure TODO

list of annotated properties 

\subsubsection{Candidate Parameter}
\subsubsection{Threshold Parameter}