%!TEX root = ../lod-group1.tex
\section{Discussion}
\label{sec_discussion}

The results shown in section \ref{sec_evaluation} give a lot of insight on the used approach.
The following section will discuss these findings.

TODO Stefan, Dominik, Tanja

Outlook SMM Comparision

Figure \ref{fig_baseline_comparison} shows the huge improvement of the discussed algorithm to the baseline approach.
This improvement can be explained by the limiting factors of the baseline approach, which have been explained in section \ref{subsec_method_matching}.


Even though the results are already good there is still room for improvements.
An analysis of the wrongly matched movies and the not found matches has yielded that the biggest possible gain is in improving the way the candidates are selected.
Since candidate selection is based mainly on the name, movies with names that are not annoted sofar are most likely not found.
Thus, further criterias should be concidered when selecting the candidates to address this problem.
However, the poor annotation quality, i.e. meaning that not many attributes are annotated, of a lot of the lesser known movies makes choosing these criterias a difficult task.


Biggest gain possible: in candidate selection
aber candidate selection is trotzdem schon gut

discuss: cartoons, evaluation

ergebnisse sind, deshalb

verbessern damit

