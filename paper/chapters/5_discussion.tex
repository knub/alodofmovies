%!TEX root = ../lod-group1.tex
\section{Discussion of Evaluation Results}
\label{sec_discussion}

The results shown in Section \ref{sec_evaluation} give a lot of insight on the used approach.
The following section will discuss these findings and give insight on possible conclusions.

Outlook SMM Comparision

Figure \ref{fig_baseline} shows the huge improvement of the discussed algorithm to the baseline approach.
This improvement can be explained by the limiting factors of the baseline approach, which have been explained in section \ref{subsec_method_matching}.
On the otherhand

Even though the results are already good there is still room for improvements.
The analysis of the wrongly matched movies and the not found matches, as listed in Section \ref{subsec_evaluation_matching} has yielded that the biggest possible gain is in improving the way the candidates are selected.
Since candidate selection is based mainly on the name, movies with names that are not annoted sofar are most likely not found.
Thus, further criterias should be concidered when calculating the pre-score for selecting candidates.
Especially a set of movies that is choosen without an influence of there name could help.
However, the movies that need to be matched, i.e. the one that do not have an annotated IMDb-id, usually have many other properties not annotated.

99,5\% precision is a really good result.
Biggest gain possible: in candidate selection
surprisingly, wird nicht besser wenn candidate set size groesser -> erklaeren

aber candidate selection is trotzdem schon gut

discuss: cartoons, evaluation

ergebnisse sind, deshalb

verbessern damit

