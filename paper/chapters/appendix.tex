%!TEX root = ../lod-group1.tex

\section{Example}
\label{appendix_example}

To illustrate the definitions for \emph{True Positives (TP)}, \emph{False Positivies (FP)}, \emph{True Negatives (TN)} and \emph{False Negatives (FN)}, the following example is given.

\begin{align*}
M &= \{LotR\textsubscript{M}, FG\textsubscript{M}, HP\textsubscript{M}\} \\
N &= \{LotR\textsubscript{N}, FG\textsubscript{N}, GoT\textsubscript{N}, BB\textsubscript{N}, HP\textsubscript{N}\},~where \\
\\
LotR &= Lord~of~the~Rings \\
FG &= Forrest~Gump \\
GoT &= Game~of~Thrones \\
BB &= Breaking~Bad \\
HP &= Harry~Potter
\end{align*}

$M$ is the current dataset. $N$ is the set of new movies, that need to be matched. The subscript indicates, from which data source a movie comes from, e.g. $LotR_N$ is the movie $LotR$ from the new movie set.
As a consequence, the gold standard would be:

\begin{align*}
gold(LotR_N) &:= LotR_M \\
gold(FG_N) &:= FG_M \\
gold(GoT_N) &:= \bot~~(not~in~the~database) \\
gold(BB_N) &:= \bot~~(not~in~the~database) \\
gold(HP_N) &:= HP_M \\
\end{align*}

Assuming the system returned the following matches:

\begin{align*}
\{ (LotR\textsubscript{N}&, LotR\textsubscript{M}), \\
(FG\textsubscript{N}&, HP\textsubscript{M}), \\
(GoT\textsubscript{N}&, \mathbf{\bot}), \\
(BB\textsubscript{N}&, HP\textsubscript{M}), \\
(HP\textsubscript{N}&, \mathbf{\bot}) \} \\
\end{align*}

Then, by the definitions given in Section \ref{subsec_evaluation_matching}, the sets would look like this:

\begin{table}[h]
\centering
\begin{tabular}{l||l|l|l|l|l}
                       & \textbf{LotR\textsubscript{N}} & \textbf{FG\textsubscript{N}} & \textbf{GoT\textsubscript{N}} & \textbf{BB\textsubscript{N}} & \textbf{HP\textsubscript{N}} \\ \hline \hline
\textbf{LotR\textsubscript{M}}  & TP                    & TN                  & TN                   & TN                   & TN                  \\ \hline
\textbf{FG\textsubscript{M}}    & TN                    & FN                  & TN                   & TN                   & TN                  \\ \hline
\textbf{HP\textsubscript{M}}    & TN                    & FP                  & TN                   & FP                   & FN                  \\ \hline
$\mathbf{\bot}$                 & TN                    & TN                  & TN                   & TN                   & TN                  \\
\end{tabular}
\caption{This shows the definitions \emph{TP}, \emph{TN}, \emph{FP} and \emph{FN}.}
\label{tab_appendix}
\end{table}

\section{Table}


\begin{longtable}{l|l|l|l|l|l}
\multicolumn{1}{c|}{\bfseries min$\_$score} & \multicolumn{1}{|c|}{\bfseries act$\_$dist} & \multicolumn{1}{|c|}{\bfseries Precision} & \multicolumn{1}{|c|}{\bfseries Recall} & \multicolumn{1}{|c|}{\bfseries F1} & \multicolumn{1}{|c}{\bfseries F0.5}
\\ \hline \hline
\endhead

\multirow{4}{*}{0.1} & 1 & 0.9938340807 & 0.8829681275 & 0.9351265823 & 0.9694881890 \\ \hhline{~-----}
& 2 & 0.9938340807 & 0.8829681275 & 0.9351265823 & 0.9694881890 \\ \hhline{~-----}
& 3 & 0.9916154276 & 0.8834661355 & 0.9344219120 & 0.9679179398 \\ \hhline{~-----}
& 4 & 0.9833610649 & 0.8829681275 & 0.9304644450 & 0.9614967462 \\ \hline
 \hline
\multirow{4}{*}{0.2} & 1 & 0.9949409781 & 0.8814741036 & 0.9347768682 & 0.9699693117 \\ \hhline{~-----}
& 2 & 0.9949494949 & 0.8829681275 & 0.9356200528 & 0.9703371278 \\ \hhline{~-----}
& 3 & 0.9943946188 & 0.8834661355 & 0.9356540084 & 0.9700349956 \\ \hhline{~-----}
& 4 & 0.9910564561 & 0.8829681275 & 0.9338951804 & 0.9673723265 \\ \hline
 \hline
\multirow{4}{*}{0.3} & 1 & 0.9954827781 & 0.8779880478 & 0.9330510717 & 0.9695336560 \\ \hhline{~-----}
& 2 & 0.9954878737 & 0.8789840637 & 0.9336154456 & 0.9697802198 \\ \hhline{~-----}
& 3 & 0.9954929577 & 0.8799800797 & 0.9341792228 & 0.9700263505 \\ \hhline{~-----}
& 4 & 0.9932659933 & 0.8814741036 & 0.9340369393 & 0.9686952715 \\ \hline
 \hline
\multirow{4}{*}{0.4}& 1 & 0.9954622802 & 0.8740039841 & 0.9307875895 & 0.9685430464 \\ \hhline{~-----}
& 2 & 0.9954751131 & 0.8764940239 & 0.9322033898 & 0.9691629956 \\ \hhline{~-----}
& 3 & 0.9954802260 & 0.8774900398 & 0.9327686607 & 0.9694102113 \\ \hhline{~-----}
& 4 & 0.9943661972 & 0.8789840637 & 0.9331218610 & 0.9689284146 \\ \hline
 \hline
\multirow{4}{*}{0.5}& 1 & 0.9959977130 & 0.8675298805 & 0.9273356401 & 0.9673478454 \\ \hhline{~-----}
& 2 & 0.9954493743 & 0.8715139442 & 0.9293680297 & 0.9679203540 \\ \hhline{~-----}
& 3 & 0.9954571266 & 0.8730079681 & 0.9302202176 & 0.9682942996 \\ \hhline{~-----}
& 4 & 0.9954674221 & 0.875        & 0.9313543599 & 0.9687913542 \\ \hline
 \hline
\multirow{4}{*}{0.6}& 1 & 0.9728531856 & 0.8745019920 & 0.9210595332 & 0.9514521023 \\ \hhline{~-----}
& 2 & 0.9725274725 & 0.8814741036 & 0.9247648903 & 0.9528423773 \\ \hhline{~-----}
& 3 & 0.9726477024 & 0.8854581673 & 0.9270072993 & 0.9538626609 \\ \hhline{~-----}
& 4 & 0.9727223131 & 0.8879482071 & 0.9284040614 & 0.9544967880 \\ \hline
 \hline
\multirow{4}{*}{0.7}& 1 & 0.9964264443 & 0.8331673307 & 0.9075128831 & 0.9588491518 \\ \hhline{~-----}
& 2 & 0.9958920188 & 0.8451195219 & 0.9143318966 & 0.9615820490 \\ \hhline{~-----}
& 3 & 0.9959159860 & 0.8500996016 & 0.9172487910 & 0.9628835740 \\ \hhline{~-----}
& 4 & 0.9959325973 & 0.8535856574 & 0.9192813087 & 0.9637876743 \\ \hline

\caption{Results for Overlap and Score}
\label{tab_overlapScore}
\end{longtable}





\begin{table}[h!]
\begin{tabular}{l|l|l|l|l}
\multicolumn{1}{c|}{\bfseries candidate$\_$set$\_$size} & \multicolumn{1}{|c|}{\bfseries Precision} & \multicolumn{1}{|c|}{\bfseries Recall} & \multicolumn{1}{|c|}{\bfseries F1} & \multicolumn{1}{|c}{\bfseries F0.5}
\\ \hline \hline
%\endhead

2    & 0.977246871445 & 0.855577689243 & 0.91237387148 & 0.950221238938 \\ \hline
4    & 0.975355969332 & 0.886952191235 & 0.92905581638 & 0.956292955326 \\ \hline
8    & 0.974040021633 & 0.896912350598 & 0.93388644024 & 0.957571246278 \\ \hline
16   & 0.973584905660 & 0.899402390438 & 0.93502459229 & 0.957685320322 \\ \hline
32   & 0.973103819258 & 0.900896414343 & 0.93560899922 & 0.957750952986 \\ \hline
64   & 0.971535982814 & 0.900896414343 & 0.93488372093 & 0.956535532995 \\ \hline
128  & 0.968415417559 & 0.900896414343 & 0.93343653251 & 0.954113924051 \\ \hline
256  & 0.966382070438 & 0.901892430279 & 0.93302421432 & 0.952756734007 \\ \hline
512  & 0.960742705570 & 0.901892430279 & 0.93038787567 & 0.948366149979 \\ \hline
1024 & 0.955168776371 & 0.901892430279 & 0.92776639344 & 0.944015846539 \\ \hline
2048 & 0.950656167979 & 0.901892430279 & 0.92563250703 & 0.940486082260 \\ \hline
4096 & 0.949161425576 & 0.901892430279 & 0.92492339122 & 0.939315352697 \\ \hline

\end{tabular}
\caption{Candidate set size}
\end{table}


\begin{table}[h!]
\begin{tabular}{l||l|l|l}
 &  \multicolumn{1}{|c|}{\bfseries Precision} &  \multicolumn{1}{|c|}{\bfseries Recall} &  \multicolumn{1}{|c}{\bfseries F0.5}
\\ \hline \hline
%\endhead

\textbf{Baseline}       & 0.5923566878980892 & 0.5925663716814159 & 0.592398612782221 \\ \hline
\textbf{MFM with id}    & 0.991807755324959 & 0.88671875 & 0.9688433632095604 \\ \hline
\textbf{MFM without id} & 0.8928571429 & 0.7142857143 & 0.8503401361 \\ \hline

\end{tabular}
\caption{Baseline}
\end{table}

%precision_baseline,precision_mfm_with_id,precision_mfm_without_id,recall_baseline,recall_mfm_with_id,recall_mfm_without_id,f0.5_baseline,f0.5_mfm_with_id,f0.5_mfm_without_id
%0.5923566878980892,0.991807755324959,0.8928571429,0.5925663716814159,0.88671875,0.7142857143,0.592398612782221,0.9688433632095604,0.8503401361

